% !TEX options=--synctex=1 --shell-escape --interaction=nonstopmode %DOCFILE%
% !TEX program=xelatex
\DocumentMetadata{}
\documentclass[openany]{book}
\InputIfFileExists{eg-config.tex}{}{}
\usepackage{amsmath,amssymb}
\PassOptionsToPackage{silent}{xeCJK}
\usepackage{xunicode-addon}
\usepackage{ctex}
\usepackage{graphicx,xcolor}
\usepackage[library={ref,box,bnf,doc={no-index-file},pdf}]{cus}
\graphicspath{{./image}{../image}{../}}
\cussetup[pdf]{pax={cus-cn},redefine}
\ifdefined\directlua
  \let\xeCJKVerbAddon\empty
\else \usepackage{xeCJKfntef}
\fi
\edef\UD{\string _}
\newcommand{\hook}{\cmd[module=hook point,type=hook point]}
\setuplayout{paper=a4,hmargin=1.7cm,top=2cm,bottom=1.5cm,
  hfoffset=0pt,nomarginpar,
  columnsep=35pt,headsep=10pt,footskip=30pt,}
\pagestyle{fancy}
\sethead [l] {{\CusTeX} --- 定义 \veta{ekeys-cmd}}
\sethead [r] {Page -- \thepage}
\setfoot {}
\setheadrulewidth {1pt}
% \usepackage{pdfpages}
\usepackage{unicode-math}
% \setmainfont{texgyrepagella}[
%   Extension      = .otf,
%   UprightFont    = *-regular,
%   BoldFont       = *-bold,
%   ItalicFont     = *-italic,
%   BoldItalicFont = *-bolditalic]
\setmainfont{TeXGyreTermesX}[
  Extension      = .otf,
  UprightFont    = *-Regular,
  BoldFont       = *-Bold,
  ItalicFont     = *-Italic,
  BoldItalicFont = *-BoldItalic,
  SlantedFont    = *-Slanted,
  BoldSlantedFont= *-BoldSlanted]
\setsansfont{texgyreheros}[
  Extension      = .otf,
  UprightFont    = *-regular,
  BoldFont       = *-bold,
  ItalicFont     = *-italic,
  BoldItalicFont = *-bolditalic]
\setmonofont{cmun}[
  Extension      = .otf,
  UprightFont    = *btl,
  BoldFont       = *tb,
  ItalicFont     = *bto,
  BoldItalicFont = *st,
  HyphenChar     = None]
\setmathfont{XITS Math}
\usepackage{array,booktabs,tabularx,makecell}
\usepackage{enumitem}
\setlist{nosep}
\setlist[enumerate,1]{leftmargin=\ccwd,itemindent=\ccwd,listparindent=2\ccwd}
\setlist[itemize,1]{label=$\smblkdiamond$,leftmargin=\ccwd,itemindent=\ccwd,listparindent=2\ccwd}
\setlist[itemize,2]{label=\textbullet,leftmargin=\ccwd,itemindent=\ccwd,listparindent=2\ccwd}
\setlist[itemize,3]{label=$\smwhtcircle$}
\usepackage[colorlinks]{hyperref}
\usepackage[numbered]{bookmark}
\usepackage{nameref,varioref,cleveref}
\usepackage{fancyvrb}
\usepackage[many,listings]{tcolorbox}
\usepackage{texhigh}
\tcbset{l3code/.style={texhigh use ctab={latex3code}}}
\SetKeys[texhigh]{
  font=\ttfamily\xeCJKsetup{CJKecglue={\hskip 0pt plus 0.08\baselineskip}},
  config-file=ekeys.texhigh.cfg,
}
\makeatletter
\THSetClassCS{cus.ekeys.doc}{{\color{red5}\bfseries#1\itshape#2}}
\THSetClassCS{cus.ekeys.l3}{{\color{red5}\bfseries#1\itshape#2}}
\THSetClassCS{cus.collectn}{{\color{red5}\bfseries#1\itshape#2}}
\THSetClassCS{cus.elkernel}{{\color{red5}\bfseries#1\itshape#2}}
\makeatother
\newcounter{example}
\newtcblisting[use counter=example, number format=\arabic, crefname={代码}{代码}]
  {examcode}[2][]{listing and text, 
  title=代码 \thetcbcounter, enhanced,
  comment={#2},
  sharp corners=downhill, arc=12pt, %skin=bicolor,
  fontupper=\linespread{1}\selectfont, left=6pt,
  colback=blue!1!white, colframe=blue!75!black,colbacklower=white,
  segmentation style={draw=blue,thick,solid},
  attach boxed title to top right={yshift=-\tcboxedtitleheight},
  boxed title style={
    colframe=blue!75!black,colback=blue!15!white,
    sharp corners=downhill,arc=12pt,
  },
  coltitle=blue!90!black, fonttitle=\bfseries,
  before skip balanced=2bp plus .5\baselineskip,
  after skip balanced=2bp plus .5\baselineskip,
  listing engine=texhigh,
  breakable,
  #1
}
\fvset{formatcom=\xeCJKVerbAddon,xleftmargin=2cm,xrightmargin=2cm,
  vspace=\smallskipamount}

\setlength{\abovecaptionskip}{3pt plus 2pt}
\setlength{\belowcaptionskip}{3pt plus 2pt}
\setlength{\intextsep}{3pt plus 5pt}
\setlength{\floatsep}{3pt plus 5pt}
\setlength{\textfloatsep}{3pt plus 5pt}

\cussetup[texbnf]{hyper,
  geometry={\leftmargin2cm \rightmargin2cm \topsep\smallskipamount}}
\crefformat{section}{第 #2#1#3 节}
\crefformat{table}{表 #2#1#3}
\def\thesection{\texorpdfstring{\S\ }{\S}\arabic{section}} 
\def\theHsection{\arabic{section}}
\setuptitle[section]{number=\thesection,format=\large\bfseries,
  beforeskip=3pt plus2pt minus2pt,afterskip=3pt plus2pt minus2pt}

\makeatletter
\renewcommand\@makefnmark{\hbox{\hskip.1em\textsuperscript
  {\ExpandArgs{f}\textcircled{\@thefnmark}}}}
\renewcommand\@makefntext[1]{\noindent\ExpandArgs{f}\textcircled{\@thefnmark}\ #1}
\newcommand\ssep{\smallskip}
\newcommand\bsep{\bigskip}

\def\maketitle{\bookmark[dest=Doc-Start,level=1]{\thesection\ 封面}%
  \begin{titlepage}%
  \let\footnotesize\small
  \let\footnoterule\relax
  \let \footnote \thanks
  \null\vfil
  \vskip 60\p@
  \begin{center}%
    {\LARGE \@title \par}%
    \vskip 3em%
    {\large
     \lineskip .75em%
      \begin{tabular}[t]{c}%
        \@author
      \end{tabular}\par}%
      \vskip 1.5em%
    {\large \@date \par}%       % Set date in \large size.
  \end{center}\par
  \vskip 2cm
  \def\@pnumwidth{1.1em}\def\@tocrmarg{1.2em plus 5pt minus 5pt}\def\@dotsep{3}%
  \SetPlainTocLevelCode{section}{\@dottedtocline{1}{0em}{.7em}{##5}{##6}}%
  \multicolplaincombinedlist[2,ragged,shorten=1cm,column-sep=2em]{}{toc}%
  \@thanks
  \vfil\null
  \end{titlepage}%
  \setcounter{footnote}{0}%
  \global\let\thanks\relax \global\let\maketitle\relax
  \global\let\@thanks\@empty \global\let\@author\@empty
  \global\let\@date\@empty \global\let\@title\@empty
  \global\let\title\relax \global\let\author\relax
  \global\let\date\relax \global\let\and\relax
}

\ExplSyntaxOn
\NewDocumentCommand \token { O{12} +m }
  {
    \mode_leave_vertical: { \cs_set_eq:NN \\ \c_backslash_str
    \dim_set:Nn \fboxsep { 2pt } \ttfamily 
    \tl_if_blank:nTF {#2}
      { 
        \textvisiblespace 
        \tl_if_empty:nT {#1} \use_none:nn \textsubscript {#1} 
      }
      {
        \str_set:Nx \l_tmpa_str {
          \bool_lazy_and:nnTF
            { \tl_if_single_token_p:n {#2} }
            { \tl_if_head_eq_catcode_p:nN {#2} \scan_stop: }
            { \cs_to_str:N #2 }
            { #2 }
        }
        \tl_replace_all:Nnn \l_tmpa_str { ~ } { \textvisiblespace }
        \tl_if_single:NTF \l_tmpa_str
          { \l_tmpa_str \tl_if_empty:nT {#1} \use_none:nn \textsubscript {#1} }
          { \fbox { \vphantom{fp}\l_tmpa_str } }
      }
  } }
\ExplSyntaxOff
\makeatother

% \raggedbottom 
\title{定义 \veta{ekeys-cmd}}
\author{雾月\quad Longaster}
\begin{document}

\enablecombinedlist
\maketitle

\section{前言}

\pkg{lt3ekeyscmd} 和 \pkg{lt3ekeysext} 这两个宏包提供了一个定义命令的新接口。
它的使用方式类似于 \pkg{ltcmd} 的 \cs{DeclareDocumentCommand},功能更加丰富的同时
也更加复杂。本文将详细介绍一下这两个宏包以及它们的前置宏包——\pkg{collectn},
这个宏包在定义获取特殊的参数的命令时是非常有用的。

以下称由 \cs{NewDocumentCommand} 系列命令定义的命令为 \veta{document-cmd},
包括它们的完整复制\footnote{即由 \cs{NewCommandCopy}、\cs{RenewCommandCopy}、\cs{DeclareDocumentCopy} 复制的。}。

首先来看看 \cs{DeclareDocumentCommand} 的用法。
\begin{syntax}
  \V\DeclareDocumentCommand <cmd> <{arg spec}> <{code}>
\end{syntax}
这 \meta{cmd} 就是要定义的命令,\meta{arg spec} 是参数说明符列表,\meta{code} 是命令的定义。

参数说明符(argument specification)是用来标识参数性质的单个字母,有的参数说明符还可能需要
一些额外的信息。“参数性质”就是告诉 \LaTeX 以什么方式获取这个参数,以及传递给实参什么内容。
例如有的参数是通过某个符号是否出现来传递给实参不同的内容,有的参数包裹在一对 \verb|[ ]| 里,
有的参数传递给实参是一些键值对,有的还需对获取的参数经过一些处理再作为实参,等等。
根据这些性质的不同,就有了各样的参数类型,不同的参数类型就是根据参数说明符(以及可能的额外信息)
来标识的。

参数可以分为必须给出的(即如果没有则会出错),和可选的(即不给出也能正常执行)。
定义 \veta{document-cmd} 可用的说明符如下:
\begin{itemize}
  \item 必须的:\texttt m、\texttt r、\texttt R、\texttt v 以及 \texttt b(仅能在定义环境时使用);
  \item 可选的:\texttt o、\texttt d、\texttt O、\texttt D、\texttt s、\texttt t、\texttt e、\texttt E。
\end{itemize}
这些参数类型在 \LaTeXe 内核中定义,可以直接使用。此外,
还有标记为过时的:
\begin{itemize}
  \item 必须的:\texttt l、\texttt u;
  \item 可选的:\texttt g、\texttt G;
\end{itemize}
它们仅能在加载 \pkg{xparse} 宏包后使用,且定义 \veta{document-cmd} 时一般不推荐使用。

除了参数说明符还有几种参数修饰符,它们可以放在说明符的前面,用于修饰这些参数类型:
\texttt +、\texttt !、\texttt =、\texttt >。

本文在此并不准备对这些参数类型和修饰符进行说明。不过,\veta{ekeys-cmd} 和 
\veta{document-cmd} 的相同参数类型其用法基本相同,且本文会对前者进行详细介绍,因此,读者
若不了解上述参数说明符也可继续阅读下去。
如若想了解上述参数类型和修饰符的用法,可参考 \file{usrguide.pdf}。

\section{复制和显示命令}

\veta{document-cmd} 除了可以使用 \cs{NewDocumentCommand} 系列命令定义外,还可
通过复制另一个 \veta{document-cmd} 得到。不过必须是“完整复制”。
\begin{syntax}
  \V\DeclareCommandCopy \meta{new cmd} \meta{old cmd}
\end{syntax}
\cs{DeclareCommandCopy} 系列命令不仅可以完整复制由 \cs{newcommand}、
\cs{DeclareRobustCommand} 定义的命令,还可以完整复制 \veta{document-cmd} 和 
\veta{ekeys-cmd}。

在终端中显示命令的定义也很简单,使用 \cs{ShowCommand}\marg{cmd} 即可,
同样支持由 \cs{newcommand}、\cs[break-at-any]{DeclareRobustCommand} 定义的命令,
以及 \veta{document-cmd} 和 \veta{ekeys-cmd}。

\section{通用命令钩子}

通用命令钩子就是形如 
\hook{cmd/\meta{cmd name}/before} 或 \hook{cmd/\meta{cmd name}/after}
的钩子。通用命令钩子无需声明即可使用。前者在命令最开始时执行,后者在命令最末尾执行。

\cs{AddToHook}、\cs{AddToHookWithArguments}、\cs{AddToHookNext}、
\cs{AddToHookNextWithArguments} 这四个命令用于给通用命令钩子添加代码。
对于没有声明的命令钩子,可以
自动为命令添加使用钩子的代码 \cs[break-at-any]{UseHookWithArguments}(本文称为自动“修补”),
同样支持由 \cs{newcommand}、\cs{DeclareRobustCommand} 定义的命令,
以及 \veta{document-cmd} 和 \veta{ekeys-cmd}。

不过自动添加 \cs{UseHookWithArguments} 并非总会执行。
\begin{itemize}
  \item 对于已经声明的命令钩子,不会自动添加 \cs{UseHookWithArguments};
  \item 对于在导言区使用的 \cs{AddToHook} 系列命令,自动修补在 \hook{begindocument} 钩子中才会执行;
  \item 对于在正文区使用的 \cs{AddToHook} 系列命令,自动修补仅执行一次,命令重定义以后不会再执行自动修补,因为已经自动声明了这个钩子。
\end{itemize}

自动修补并不总是有效,有时甚至会导致严重的错误。例如:
\begin{Verbatim}
\newcommand\fancybox{\@ifnextchar({\@fancybox}{\@fancybox(5cm)}}
\def\@fancybox(#1)#2{\fbox{\parbox{#1}{#2}}}
\end{Verbatim}
倘若想在 \verb|\fancybox| 后面添加一些代码,使用
\begin{Verbatim}
\AddToHook{cmd/fancybox/after}{<code>}
\end{Verbatim}
结果是 \verb|\fancybox| 变成了:
\begin{Verbatim}
\newcommand\fancybox{\@ifnextchar({\@fancybox}{\@fancybox(5cm)}%
  \UseHookWithArguments{cmd/fancybox/after}{0}}
\end{Verbatim}
会导致严重的错误。

因此若要使用命令钩子,最佳实践是在定义命令时就加上 \cs{UseHookWithArguments}:
\begin{Verbatim}
\newcommand\fancybox{\@ifnextchar({\@fancybox}{\@fancybox(5cm)}}
\def\@fancybox(#1)#2{\fbox{%
  \UseHookWithArguments{cmd/fancybox/before}{2}{#1}{#2}%
  \parbox{#1}{#2}%
  \UseHookWithArguments{cmd/fancybox/after}{2}{#1}{#2}}}
\end{Verbatim}

当然,使用了 \cs{UseHookWithArguments} 后需要更长的执行时间,具体如何做就需要自行斟酌了。

关于钩子和命令钩子的具体用法,见 \file{lthooks-doc.pdf} 和 \file{ltcmdhooks-doc.pdf}

\section{控制序列获取其需要的参数}

根据 \TeX 中控制序列获取其需要的参数的方式总体可分为两类:一类是 \TeX 语法规定的,另一类
是定义宏时设置的。前者基本固定,只需阅读 \TeX 引擎的文档即可知晓;而后者则可以十分灵活。

在 \LaTeX 格式下,命令获取参数的规则有许多一致的地方,例如命令后面跟着一个可选的 \verb|*|,
或者是跟着可选的一个或多个由 \verb|[ ]| 包裹的实参,最常见的就是由一对 \verb|{ }| 包裹的实参。

而由一对 \verb|{ }| 包裹的“实参”从可使用的 \verb|{ }| 上看又有多种情况:
\begin{enumerate}
  \item\label{item:single} 是除了显式的左括号(记为 \BNFanchor{left brace})、右括号(记为 \BNFanchor{right brace})和空格\footnote{显式字符就是有唯一类别码的普通字符,隐式字符就是被 \tn{let} 为显式字符的控制序列或活动字符。显式的左、右括号和空格就是类别码分别为 1、2、10 的字符。}之一的单个记号(token),此时无需加上 \BNFN{left brace} 和 \BNFN{right brace};
  \item[1$'$.]\label{item:eler} 由 \BNFN{left brace} 和 \BNFN{right brace} 包裹,且它们之间的 \BNFN{left brace} 和 \BNFN{right brace} 数量保持平衡\footnote{所谓“保持平衡”即数量一致。};
  \item\label{item:ler} 由 \BNFT{\{} (表示显式的或隐式的类别码\footnote{“类别码”是一个字符记号拥有的属性,在 \XeTeX 和 \LuaTeX 中,每一个字符都有唯一的字符码(就是它的 Unicode 编码)和类别码。可以修改字符的类别码为任何有效的值。\pdfTeX、\XeTeX、\LuaTeX 中类别码的有效值和它们的含义见\cref{tab:catcode}。}为 1 的字符)以及 \BNFN{right brace} 包裹,且它们之间的 \BNFN{left brace} 和 \BNFN{right brace} 数量保持平衡;
  \item\label{item:sler} 由某一特定的类别码为 1 的显式字符和 \BNFN{right brace} 包裹,且它们之间的 \BNFN{left brace} 和 \BNFN{right brace} 数量保持平衡;
  \item\label{item:silsir} 由某一特定的隐式左、右括号包裹,且它们之间的 \BNFN{left brace} 和 \BNFN{right brace} 数量保持平衡;
  \item\label{item:lr} 由 \BNFT{\{} 和 \BNFT{\}} 包裹。
\end{enumerate}
其中 第 \ref{item:single} 和 \ref{item:eler}$'$ 可认为是一类。
\LaTeX 格式中,大部分命令都使用第 \ref{item:single} 类(含 1$'$,如无特别说明,以后同),
除了那些暴露的 \TeX 原语。

\LaTeXiii 则区分 \ref{item:single} 和 1$'$,前者为 \texttt N 类型,后者为 \texttt n 类型。
实际使用时,尽管在某些情况下它们可以混用,但仍然有例外\footnote{如 \cs{token_to_str:N}、\cs{token_to_meaning:N} 的参数可以是任意记号,\cs{tl_to_str:n}、\cs{exp_not:n} 可以是第 1$'$ 或第 \ref{item:ler} 类,但不能是第 \ref{item:single} 类。},因此最好严格按照类型指定的用法使用,至少这样不会出错。除非你了解这些命令的定义并且确信它们不会更改。

\section{\pkg{lt3ekeyscmd} 和 \pkg{lt3ekeysext} 简介}

\pkg{lt3ekeyscmd} 提供定义 \veta{ekeys-cmd} 的主要接口,\pkg{lt3ekeysext}
扩展了前者的功能,提供了几个参数修饰符和额外的参数说明符,并且支持自定义参数获取方式。
加载 \pkg{lt3ekeysext} 对 \veta{ekeys-cmd} 的使用上没有任何影响,且功能更强大,
推荐直接加载 \pkg{lt3ekeysext}。本文示例代码也建立在加载 \pkg{lt3ekeysext} 的基础上。

可通过 \cs{DeclareEKeysCommand} 来定义 \veta{ekeys-cmd}:
\begin{syntax}
  \V\DeclareEKeysCommand   \meta{cmd} \marg{arg spec} \marg{code}
  \V\DeclareEKeysCommand * \meta{cmd} \marg{arg spec} \marg{code}
\end{syntax}
\cs{DeclareEKeysCommand} 和 \cs{DeclareDocumentCommand} 使用上几乎没有区别,
只是前者还支持星号可选参数,它的功能后面再说明。
\cs{DeclareEKeysCommand} 也可以写为 \cs{ekeysdeclarecmd}。

以下列出 \veta{ekeys-cmd} 可用的参数说明符,其中在 \pkg{lt3ekeyscmd} 中定义的有:
\begin{itemize}
  \item 必须的:\texttt m、\texttt r、\texttt R、\texttt l、\texttt u、\texttt U、\texttt v;
  \item 可选的:\texttt o、\texttt O、\texttt d、\texttt D、\texttt s、\texttt t、\texttt p、\texttt P、\texttt k、\texttt t、\texttt T、\texttt e、\texttt E、\texttt w、\texttt W、\texttt g、\texttt G。
\end{itemize}
在 \pkg{lt3ekeysext} 中定义的有:
\begin{itemize}
  \item 必须的:\texttt c、\texttt C;
  \item 自定义的:\texttt K;
  \item 预处理指示符:\texttt ?;
  \item 参数修饰符:\texttt\&、\texttt\#、\texttt @。
\end{itemize}

其中,\veta{document-cmd} 标记为过时的 \texttt g、\texttt G、\texttt l、\texttt u 
参数类型,\veta{ekeys-cmd} 是直接支持的,不过其行为略有不同。在此,我们不讨论
“应该做什么、不应该做什么”,而注重讨论“能做什么”和“怎么做”。这也是 
\pkg{lt3ekeyscmd} 和 \pkg{lt3ekeysext} 的作者编写这两个宏包所遵循的原则。
至于要不要这么做,则交由读者自行考量。

这其中有两个参数类型使用了同一个参数说明符,但使用时会有区别,因此即使如此也不会混淆。

\veta{ekeys-cmd} 的参数说明符与 \veta{document-cmd} 相同的部分,其用法和作用基本相同,
不过也有例外的情况,在此先做概述,待后文详细说明:
\begin{itemize}
  \item 所有参数都是 \tn{long},即都可以包含 \token{par}\footnote{\token{par} 表示一个控制序列,它的名字为 \texttt{par}。};
  \item 向后寻找可选参数时,忽略空格。但这未来可能会更改;
  \item 带有定界符的参数类型,如 \texttt r、\texttt d,默认情况下是不会处理嵌套的定界符的,但可以在该参数说明符的前面加上 \texttt @,这样就会处理嵌套的定界符了。
  \item 有默认值的参数类型,默认情况下是不会处理用 \verb|#1| 等引用其它实参的,但可以使用带星号的 \cs[break-at-any]{DeclareEKeysCommand},这样就和 \veta{document-cmd} 一样了。不过默认值之间不能循环引用。\footnote{关于默认值,另外还有一种情况与 \veta{document-cmd} 不同:当默认值引用了其它实参,且此参数的值为特殊的 -NoValue- 标记时,\veta{ekeys-cmd} 不会进行引用替换,而 \veta{document-cmd} 会。由于该特殊的标记是内部值,用户一般不能输入,但若放在另一个 \veta{document-cmd} 或 \veta{ekeys-cmd} 则是可能的。}
  \item \texttt s 和 \texttt t 参数类型及带定界符的参数类型,还可以通过加上 \texttt\& 修饰符,使得其实参为等价的原始输入,而不是布尔值或移除掉定界符的结果。
  \item \texttt e 和 \texttt E 参数类型,在 \veta{document-cmd} 其中的 \meta{tokens} 是不能重复的,但 \veta{ekeys-cmd} 却可以。
  \item \texttt g 和 \texttt G 参数类型,在寻找其参数时,左括号既可以是显式的,又可以是隐式的;且若没有发现左括号,而发现的是 \tn{relax}(或 \tn{ifx} 判断与之相等),则会移除这个 \tn{relax} 后再继续寻找之后的参数。这样在实际使用时,可以用 \tn{relax} 阻止 \texttt G 参数获取实参,又不会干扰其它参数获取实参;
  \item \texttt{v} 参数类型,对于 \veta{document-cmd},其实参的类别码为 10、12、13 之一,而 \veta{ekeys-cmd} 其实参所含字符的类别码的可能值为 10、12、13(在 \pupTeX 引擎下,仅 Unicode 编码小于等于 255 的字符有此特性);且 \texttt{\^{}\^{}M} 的类别码为 13;
  \item “处理器” vs. “预处理器”。与 \veta{document-cmd} 对偶的,\veta{ekeys-cmd} 使用的是预处理器机制。且 \veta{document-cmd} 的处理器不能是 \veta{document-cmd},换句话说,\veta{document-cmd} 不能嵌套。\veta{ekeys-cmd} 则没有这个限制。
\end{itemize}

除了直接定义命令外,\pkg{lt3ekeyscmd} 还支持不使用 \veta{ekeys-cmd} 而直接收集参数:
\begin{syntax}
  \V\DeclareEKeysCollector   \meta{collector cmd} \marg{arg spec}
  \V\DeclareEKeysCollector * \meta{collector cmd} \marg{arg spec} \meta{collect to cmd} \marg{do code}
  \V\ekeyscollectargs   \meta{collect to cmd} \meta{ekeys-cmd} \marg{do code} \meta{args}
  \V\ekeyscollectargs * \meta{collect to cmd} \marg{arg spec} \marg{do code} \meta{args}
\end{syntax}
这个命令根据 \meta{ekeys-cmd} 获取参数的方式,或指定的 \meta{arg spec} ,
从 \meta{args} 收集参数并保存到 \meta{collect to cmd} 里,然后执行 \meta{do code}。
使用 \cs{DeclareEKeysCollector} 定义的命令称为 \veta{ekeys-collector}。

其中,使用 \verb|\DeclareEKeysCollector| 和 \verb|\ekeyscollectargs*| 时,
\meta{arg spec} 的数量不限,可以获取超过 9 个参数。

从执行同一作用的命令的执行时间来看,\veta{document-cmd} 长于 \veta{ekeys-cmd} 
长于 \veta{ekeys-collector} 约等于使用不带星号的 \cs{ekeyscollectargs} 直接收集参数。

\begin{syntax}
\V\ekeyscollectorarg \marg{arg number}
\end{syntax}
获取第 \meta{arg number} 个参数。可以用于上面四个命令的 \meta{do code} 中。
会在值不存在时返回 \cs{q_no_value},可使用 \cs{IfQuarkNoValueTF} 判断是否为该值。
\cs{IfNoValueTF} 用于判断是否为特殊的 -NoValue- 标记,它是用来判断可选参数是否有值的。

\section{常用的参数类型——\texttt m、\texttt R、\texttt D、\texttt t}\label{sec:fq-used}

本节介绍最为常用的参数类型:\texttt m、\texttt r、\texttt R、\texttt o、\texttt O、
\texttt d、\texttt D、\texttt s、\texttt t。

对于一个完整的参数类型 \meta{full spec},
构成为 \meta{spec} 或 \meta{spec}\marg{default},其中 \meta{spec} 为
\meta{spec name}\meta{args}。\meta{spec name} 就是标识参数类型的字母,
\meta{args} 是参数类型需要的参数,\meta{default} 也可算作是参数,不过这里单独区分它们。
就是说,\meta{full spec} 为 \meta{spec name}\meta{args or none}\marg{default or none}。

一般情况下,如未特别说明,显式或隐式的空格、左括号、右括号、宏变量字符(一般为 \verb|#|)
都不能包含在说明符的 \meta{args} 里。

\texttt m 类型的参数获取的实参是第 \ref{item:single} 类带 \verb|{ }| 的。

\texttt r、\texttt R、\texttt o、\texttt O、\texttt d、\texttt D 通过是否给出
指定的定界符来判断是否给出此实参。对于前两个,如果没有给出实参就会出错。
大小写的区别是,大写的说明符用法为 \meta{spec}\marg{default},即还需给说明符一个参数,
表示如果没有给出实参,则用此作为实参。
而小写的说明符设置 \meta{default} 为一个特殊的标记:\UseName{c_novalue_tl},
可以用 \cs{IfNoValue(TF)} 和 \cs{IfValue(TF)} 判断实参是否为此标记。

对于 \texttt R 和 \texttt D(以及对应的小写。若大小写的功能类似,只有
带与不带 \meta{default} 的区别,则只写大写,以后同);\meta{spec} 为
\meta{spec name}\meta{token_1}\meta{token_2}。
对于 \texttt O,\meta{spec} 为 \meta{spec name},相当于 \verb|D[]|。
\meta{spec name} 就是这几个说明符之一,
\meta{token_1} 和 \meta{token_2} 是单个字符或控制序列。
如 \verb|R(){NaN}|、\verb|o|、\verb|O{NaN}| 等都是正确的。

这几类带定界符的参数,默认情况下是不支持嵌套的,不过可以在它们前面加上 \texttt @ 修饰符,
来标记此参数需要处理嵌套的情形。并且判断定界符是否出现只会检查此定界符对的左定界符是否出现,
若左定界符出现了而右定界符没有出现,则会出错。

对于 \texttt s 和 \texttt t 通过是否给出指定的字符或控制序列来给出 \cs{BooleanTrue}
或 \cs{BooleanFalse},可通过 \cs[break-at-any]{IfBoolean(TF)} 判断为何者。
\meta{spec} 为 \verb|t|\meta{token},\texttt s 相当于 \verb|t*|。

对于 \meta{token} 的判断,是很严格的。如果是一个字符,则字符码和类别码都必须一致;
如果是一个控制序列,则是控制序列的名字一致。
在通过是否给出指定的记号来判断是否给出实参时,所有参数类型都遵循这个规则。

上面这些参数类型,除了 \texttt m 外,支持加上 \texttt\& 修饰符,这样实参为等价的原始输入。
参数的默认值\emph{不会}自动加上定界符。如果一个命令有带定界符的参数,而它的定义使用了
也有带定界符参数的命令,使用此前缀能减少大量的 \cs{IfValue(TF)} 判断。如:
\begin{Verbatim}
\DeclareEKeysCommand \mycaption { & s &@ O{} m } {<some code>\caption#1#2{#3}}
\end{Verbatim}
如果出现了 \verb|*| 或 \verb|[]| 可选参数,那么会 \verb|#1| 为 \verb|*|,
而 \verb|#2| 为 \verb|[{args}]|,这样省略了大量重复的 \verb|\If..| 判断。

由于 \veta{ekeys-cmd} 不支持 \texttt= 修饰符,且若设置了 \texttt\& 修饰符,那么会自动
在定界符中间加上一对 \verb|{}|,这会干扰 \veta{document-cmd} 判断其参数是否为键值对。
读者需注意这种情况。

\begin{examcode}{}
\DeclareEKeysCommand \faa { &s O{} &d() m } {\detokenize{[#1|#2|#3|#4]}}
\DeclareEKeysCommand \fbb { s @O{} @d() m } {\detokenize{[#1|#2|#3|#4]}}
\DeclareEKeysCommand \fcc { s @o @d() m } 
  {\IfBooleanTF{#1}{starred}{non-starred}, %
    \IfValueTF{#2}{b valued}{b no-valued}, %
    \IfNoValueTF{#3}{p no-valued}{p valued}, %
    \{#4\}}
\ttfamily\obeylines
\faa * ({paren(p)}) m; \faa [{bracket[b]}] {mandatory};
\fbb * (paren(p)) m;   \fbb [bracket[b]] {mandatory};
\fcc * (paren(p)) m;   \fcc [bracket[b]] {mandatory};
\end{examcode}

使用 \cs{ekeyscollectorarg} 或 \cs{tl_item:Nn} 即可获取 \veta{ekeys-collector} 
保存的参数的值。

\begin{examcode}{}
\DeclareEKeysCollector \faa { m m @o @d() m m m m m m m }
\faa\tmp{\meaning\tmp.} 12 [[B]](b(b)) 56789ABCD.
\faa\tmp{\edef\tmp{\ekeyscollectorarg{10}}\meaning\tmp.} % 把第10个值保存到 \tmp
  12 [[a]](b(b)) 56789ABCD.

\DeclareEKeysCollector*\fbb { m m @o @d() m m m m m m m } \tmp {\meaning\tmp.}
\fbb 12 [[B]](b(b)) 56789ABCD.
\end{examcode}

\section{以纯文本的形式读取参数——\texttt v}

\texttt{v} 以纯文本的形式读取参数,它要读取的参数不能做为另一个命令的参数
(不能是 tokenized)。它的实参所含字符的类别码的可能值为 10、12、13
(在 \pupTeX 引擎下,仅 Unicode 编码小于等于 255 的字符有此特性)。
且 \texttt{\^{}\^{}M} 的类别码为 13,默认情况下,它导致换行。

\begin{examcode}{}
\DeclareEKeysCommand \faa { v } {开始#1结束}
\faa {@!#$ab`";}
\faa |@!#$ab`";|
\end{examcode}

\section{获取某些记号之前的内容——\texttt l、\texttt u、\texttt U}

本节介绍的是可获取某些(某个)记号之前的所有内容的参数类型:\texttt l、\texttt u、\texttt U。

\texttt l 的 \meta{full spec} 为 \verb|l|。它获取的是字符码为 123,类别码为 1 
的记号(也就是通常的左括号 \verb|{|)之前的所有内容。\iffalse}\fi % for latex-workshop
如果不存在这样的左括号,那么将会出错。

\texttt u 的 \meta{full spec} 为 \verb|u|\marg{tokens_1},
\texttt U 的 \meta{full spec} 为 \verb|U|\marg{tokens_1}\marg{tokens_2}。
它们获取的是 \meta{tokens_1} 之前的所有内容,并移除 \meta{tokens_1}。
\texttt U 还会在移除 \meta{tokens_1} 之后留下 \meta{tokens_2}。

\texttt u 和 \texttt U 支持使用 \texttt\# 修饰符来加快执行速度。

\begin{examcode}{}
\DeclareEKeysCommand \faa { u{\relax\empty} l } {\detokenize{[#1|#2]}}
\DeclareEKeysCommand \fbb { # U{jk}{lm} l } {\detokenize{[#1|#2]}}
\ttfamily\obeylines
\faa a list tokens \relax end \relax\empty {?};
\fbb a b c i j k jk{?};
\end{examcode}

\section{判断记号是否出现——\texttt t、\texttt p、\texttt P、\texttt k、\texttt T}

本节介绍的是用来判断记号是否出现的参数类型:\texttt s、\texttt t、\texttt p、\texttt P、
\texttt k、\texttt t、\texttt T。
其中 \texttt s 和 \texttt t 已经在\cref{sec:fq-used}介绍过了。

\texttt p 的 \meta{full spec} 为 \verb|p|\marg{tokens},
\texttt P 的 \meta{full spec} 为 \verb|P|\marg{tokens}\marg{indexed}。
它们用来判断 \meta{tokens} 中的某一个记号是否出现,如果出现了,
对于 \texttt p,其实参为该记号在 \meta{tokens} 中的位置,若没有出现则实参为 0;
对于 \texttt P,其实参为用该记号在 \meta{tokens} 中的位置索引 \meta{indexed},
\meta{indexed} 的长度为 \meta{tokens} 的长度加一,即 \texttt P 的完整 
\meta{full spec} 为:
\verb|P| \texttt{\{\meta{token_1}\meta{token_2}...\meta{token_n}\}
\{\marg{entry_0}\marg{entry_1}\marg{entry_2}...\marg{entry_n}\}}。
它们支持 \texttt\# 修饰符。\texttt p 支持 \texttt\& 修饰符。

\texttt k 的 \meta{full spec} 为 \verb|k|\marg{keyword},用来判断关键字 \meta{keyword}
是否出现。为了判断 \meta{keyword} 是否存在,它会自动展开后面的内容。
\meta{keyword} 被转化为普通字符,且忽略大小写和类别码\footnote{严格地来说,“转化为普通字符”是
先对 \meta{keyword} 执行 \tn{detokenize}(\cs{tl_to_str:n}),然后把
字符码为 32、类别码为 10 的字符替换为字符码为 32、类别码为 12 的字符。
“忽略类别码”不会忽略类别码为 1、2、10、13 的字符,它们会终止展开和判断。}。
实参为 \cs{BooleanTrue} 和 \cs{BooleanFalse} 之一。它支持 \texttt\& 修饰符。

\texttt t 和 \texttt T 的 \meta{spec} 为 \meta{spec name}\marg{paired tokens},
用于判断多个定界符对 \meta{paired tokens} 中的某一对是否出现,\texttt T 还可设置默认值。
它们占用 2 个参数,前一个为该字符对在 \meta{paired tokens} 中的位置,若没有出现则为 0。
后一个实参为定界符中间的内容,若没有则为设置的默认值。
这些定界符每对的第二个还可以为空或空格。每对第一个必须不同,第二个可以相同。
它们支持 \texttt\#、\texttt @、\texttt\& 修饰符,且使用 \texttt\& 时会自动使用 \texttt\#。

注意到 \texttt t 说明符有两个用法,一个为 \verb|t|\meta{token},另一个为 
\verb|t|\marg{paired tokens},它们是完全不同的类型。\meta{token} 只能有一个记号,
而 \meta{paried tokens} 的内容是成对存在的,至少有 2 项。

\begin{examcode}{}
\DeclareEKeysCommand \faa { p{*+.} k{shipout} } {\detokenize{[#1|#2]}}
\DeclareEKeysCommand \fbb { P{*+.}{{}*+.} &k{shipout} }{\detokenize{[#1|#2]}}
\ttfamily\obeylines
\def\ipout{ipout}
\faa + shipout; \faa * SHIPOUT; \faa sh\ipout; \faa ;
\fbb + shipout; \fbb * SHIPOUT; \fbb sh\ipout; \fbb ;
\end{examcode}

当 \texttt T 类型使用 \texttt\# 前缀且有控制序列作为定界符时,定义 \veta{ekeys-cmd} 时 
\tn{escapechar} 的值必须和使用这个 \veta{ekeys-cmd} 命令时的值一致。一般情况下总是一致的。

\begin{examcode}{}
\DeclareEKeysCommand \faa { t{ () [] *{} } } {\detokenize{{#1|#2}}}
\DeclareEKeysCommand \fbb { @& T{ () [] *{} }{NaN} } {\detokenize{{#1|#2}}}
\DeclareEKeysCommand \fcc { @& t{ [] \a\b *{ } } } {\detokenize{{#1|#2}}}
\ttfamily\obeylines
\faa [{bracket[b]}]; \faa * m; \faa *{at}; \faa ;
\fbb [bracket[b]];   \fbb * m; \fbb *{at}; \fbb ;
\fcc [bracket[b]];   \fcc \a a\a r\b\b \fcc *aa*a b ; \fcc ;
\end{examcode}

\section{可以以任意顺序给出定界符的参数类型——\texttt W}

本节介绍的参数类型和 \texttt D、\texttt T 类似,但不同的是它们可以以任意顺序匹配多个定界符对:
\texttt e、\texttt E、\texttt w、\texttt W。

\texttt e 和 \texttt E 的 \meta{spec} 为 \meta{spec name}\marg{tokens}。\texttt E 
可以为 \meta{tokens} 中的每一个都设置默认值。\texttt E 类型的完整形式为 
\verb|E| \texttt{\{\meta{token_1}\meta{token_2}...\meta{token_n}\}
  \{\marg{default_1}...\marg{default_2}...\marg{default_n}\}}。它们占用 $n$ 个参数。
在获取参数时 \meta{token_i} 可以以任意顺序出现,
但作为实参时仍然按照 \meta{tokens} 所给的顺序。
例如 \verb|\foo| 的参数为 \verb|e{_^}|,那么 \verb|\foo _a^b|,\verb|\foo ^b_a| 皆可。

仍然要提及的是,\meta{tokens} 的匹配是严格的,即字符码和类别码都必须相同。

\meta{token_i} 和 \meta{token_j}($i\ne j$)可以相同,匹配参数时,已经用过的 \meta{token}
不会由于后面的出现了相同 \meta{token} 而更新其实参。

\texttt w 和 \texttt W 的 \meta{spec} 为 \meta{spec name}\marg{paired tokens}。\texttt W
可以为 \meta{paired tokens} 中的每一对都设置默认值。\texttt W 类型的完整形式为
\verb|W| 
\texttt{\{\meta{token_{11}}\meta{token_{12}}...\meta{token_{n1}}\meta{token_{n2}}\} 
  \{\marg{default_1}...\marg{default_2}...\marg{default_n}\}}。它们占用 $n$ 个参数。
在获取参数时每对定界符可以以任意顺序出现。但作为实参时仍然按照 \meta{paired tokens} 中定界符对
所给的顺序。相同的定界符对可以重复出现。定界符的左边相同时,右边也必须相同。
定界符的右边可以为空或空格,但左边不可以。

\texttt e、\texttt E、\texttt w 都是 \texttt W 的特例。它们都支持 
\texttt\#、\texttt @、\texttt\& 修饰符,且使用 \texttt\& 时会自动使用 \texttt\#。
当使用 \texttt\# 前缀且有控制序列作为定界符时,定义 \veta{ekeys-cmd} 时 
\tn{escapechar} 的值必须和使用这个 \veta{ekeys-cmd} 命令时的值一致。一般情况下总是一致的。

\begin{examcode}{}
\catcode`\^=7 \catcode`\_=8 
\DeclareEKeysCommand \faa { e{_^_} } {\detokenize{[#1|#2|#3]}}
\DeclareEKeysCommand \fbb { &e{_^_} } {\detokenize{[#1|#2|#3]}}
\DeclareEKeysCommand \fcc { @w{ [] () [] } } {\detokenize{{#1|#2|#3}}}
\DeclareEKeysCommand \fdd { @&w{ [] *{ } [] } } {\detokenize{{#1|#2|#3}}}
\DeclareEKeysCommand \fee { @&w{ [] *{ } \a\b } } {\detokenize{{#1|#2|#3}}}
\ttfamily\small\obeylines
\faa _a _{abc} ^e; \faa ^{eee} _a; \faa ;
\fbb _a _{abc} ^e; \fbb ^{eee} _a; \fbb ;
\fcc (paren(p)) [bracket[b]] [B[B]]; \fcc [bracket[b]]; \fcc ;
\fdd *hhh*h j [bracket[b]] [{bracket[}]; \fdd *jj ;
\fee \a zmg\a ?\b\b [[?]];
\end{examcode}

上述几节提到的参数说明符的用法都比较简单,读者能很快掌握。
而接下来的几个小节会介绍更加“高级”同时也更为复杂的参数类型:
\texttt c、\texttt C、\texttt K 以及预处理指示符 \texttt ?。

\section{将参数保存到寄存器的类型——\texttt c 和 \texttt C}

从本小节开始就有一定的难度了,读者需要有 \TeX 和 \LaTeXiii 宏编程的基础知识。

\texttt c 的用法为 \verb|c|\marg{collect spec}。

\texttt C 的用法为 \verb|C|\marg{allocated collect spec},\meta{allocated collect spec}
为 \meta{register}\meta{collect spec}。\meta{register} 是一个已经用诸如
\tn{newcount}、\cs{int_new:N} 等分配了的寄存器。

\meta{collect spec} 有 8 类用法:
\begin{itemize}
  \item[\texttt i] 为 \texttt{count}(\texttt{int})寄存器赋值;
  \item[\texttt d] 为 \texttt{dimen}(\texttt{dim})寄存器赋值;
  \item[\texttt s] 为 \texttt{skip} 寄存器赋值;
  \item[\texttt m] 为 \texttt{muskip} 寄存器赋值;
  \item[\texttt t] 为 \texttt{toks} 寄存器赋值;
\end{itemize}
以上 5 种,其实参为所分配的寄存器,而不是它们所保存的值。
实参作为 \tn{the} 的参数,或作为 \LaTeXiii 的 \texttt V 变体的参数都能获得它们保存的值。
\footnote{这五类寄存器,都有相应的 \cs[no-index]{\meta{type}def} 原语。因此在 \meta{allocated collect spec} 中可根据 \meta{register} 自动判断保存的类型。对于这 5 种寄存器,\texttt C 参数类型只需给出 \meta{register} 即可。如 \texttt C\tn{@tempdima} 或 \texttt C\cs{l_tmpa_dim}。}
\begin{itemize}
  \item[\texttt b] \texttt b\meta{spec},保存到水平盒子中。\meta{spec} 可以为:
  \begin{itemize}
    \item \texttt{*},表示在获取参数时可以动态调整水平盒子的宽度;
    \item \oarg{width}\oarg{pos},方括号定界的为可选参数,正如 \tn{makebox} 的前两个可选参数,分别设置盒子的宽度和水平对齐方式。
  \end{itemize}
  \item[\texttt w] \texttt w\meta{spec},为一个固定宽度的盒子赋值,类似于把盒子放在 \env{minipage} 里。\meta{spec} 可以为:
  \begin{itemize}
    \item \marg{width},盒子的宽度。盒子是一个垂直盒子,可以用 \tn{vsplit} 或 \cs{vbox_set_split_to_ht:NNn} 分割;
    \item \oarg{vpos}\oarg{height}\oarg{inner pos}\marg{width},方括号定界的为可选参数,正如 \env{minipage} 的参数。盒子是一个水平盒子。若没有可选参数,则和上一个用法一样。
  \end{itemize}
  \item[\texttt v] \texttt v\meta{spec},为一个设置了最长宽度的盒子赋值,类似于把盒子放在 \env{varwidth} 里。\meta{spec} 的用法和作用与 \texttt w 一样。
\end{itemize}
以上这些用法的参数(可选、必须参数)之间不能有任何空格。
且需注意 \meta{spec} 自身是不能有 \verb|{}| 的,花括号的添加与否由具体用法决定。
它们的实参为所分配的盒子。可作为 \tn{box},\cs{box_use:N} 等命令的参数。
\footnote{对于盒子来说,并没有 \tn{boxdef} 原语,分配新的盒子寄存器是用的 \tn{chardef} 或 \tn{mathchardef}。因此,在 \meta{allocated collect spec} 是根据 \meta{register} 是否为 \tn{chardef} 或 \tn{mathchardef} token 来判断是否是保存到盒子的。}

\begin{examcode}{}
\DeclareEKeysCommand \faa { c{d} ci } {Length: \the#1; Number: \the#2;}
\newdimen\fbbtempdim \newcount\fbbtempint
\DeclareEKeysCommand \fbb { C{\fbbtempdim} C\fbbtempint }
  {Length: \the#1; Number: \the#2;}
\ttfamily\obeylines
\faa 12pt 19 |\faa\dimexpr 12pt+8pt-10pt\relax \inteval{19+3-12} |
\fbb 12pt 19 |\fbb\dimexpr 12pt+8pt-10pt\relax \inteval{19+3-12} |
\end{examcode}

\begin{examcode}{}
\DeclareEKeysCommand \faa { cb c{b*} } {\fbox{\box#1} \fbox{\box#2}}
\newbox\fbbtempboxa \newbox\fbbtempboxb
\DeclareEKeysCommand\fbb{C\fbbtempboxa C{\fbbtempboxb b*}}{\fbox{\box#1} \fbox{\box#2}}
\faa {好的} to 3cm{好\hfil 的}\quad \fbb {好的} to 3cm{好\hfil 的}
\end{examcode}

\begin{examcode}{}
\DeclareEKeysCommand \faa { c{w{3em}} c{v{3em}} } {\fbox{\box#1} \fbox{\box#2}}
\DeclareEKeysCommand \fbb { c{w[c]{3em}} c{v[c]{3em}} } {\fbox{\box#1} \fbox{\box#2}}
\DeclareEKeysCommand \fcc { c{w[c][9ex]{3em}} c{v[c][9ex]{3em}} } 
  {\fbox{\box#1} \fbox{\box#2}}
\DeclareEKeysCommand \fdd { c{w[c][9ex][t]{3em}} c{v[c][9ex][t]{3em}} } 
  {\fbox{\box#1} \fbox{\box#2}}
基准。
\faa {好的\par 吗?} {不好\par 吗?}
\fbb {好的\par 吗?} {不好\par 吗?}
\fcc {好的\par 吗?} {不好\par 吗?}
\fdd {好的\par 吗?} {不好\par 吗?}
\end{examcode}

\section{可自定义参数获取方式的类型——\texttt K}

如果上面的小节介绍的参数获取方式还不能满足你的需要,\veta{ekeys-cmd} 还支持自定义参数获取方式
——就是使用 \texttt K 类型的参数。\texttt K 的用法为 \verb|K|\marg{scanner-and-args},其中
\meta{scanner-and-args} 为
\begin{itemize}
  \item \marg{scanner}\meta{args},\meta{scanner} 带有括号,则它的参数可以不带有括号;
  \item \meta{scanner}\marg{arg_1}\meta{extra args},带有一个或多个参数,若 \meta{scanner} 不带括号,则第一个参数必须有括号;
  \item \meta{scanner},不带任何参数的扫描器(“扫描器”就是自定义的参数获取方式);
\end{itemize}
以纯文字描述就是:移除 \meta{scanner-and-args} 的首尾空格后,
若它以一对 \verb|{ }| 开始,则这个花括号里的内容
就是 scanner,之后的内容移除掉两端的空格后就是额外的参数;
否则,就是 scanner 就是第一个花括号之前的内容(移除掉两端的空格),
额外的参数就是第一个花括号及其之后的记号;否则,没有任何花括号,就只有 scanner,没有参数。
如若 \meta{scanner-and-args} 为 \verb*| {a b } [v] j |,则 scanner 为 \verb*|a b |,
参数为 \verb*|[v] j|;
若 \meta{scanner-and-args} 为 \verb*| fo o {da} [b ] |,则 scanner 为 
\verb*|fo o|,参数为 \verb*|{da} [b ]|。

有几个预定义的扫描器:
\begin{itemize}
  \item \texttt{norelax},移除一个 \tn{relax},在向后寻找这个 \tn{relax} 时\emph{忽略}空格;
  \item \texttt{norelax!},移除一个 \tn{relax},在向后寻找这个 \tn{relax} 时\emph{不忽略}空格;
  \item \texttt ?\meta{preprocessor spec},参数预处理器。\meta{preprocessor spec} 用法为:
  \begin{itemize}
    \item \marg{ekeys-cmd arg spec},使用 \meta{ekeys-cmd arg spec} 获取后面的参数,转化为带花括号的标准参数;
    \item \marg{ekeys-cmd arg spec}\marg{scanner code},使用 \meta{ekeys-cmd arg spec} 获取后面的参数,并用 \meta{scanner code} 替换这些参数。\meta{scanner code} 必须包含 \cs{ekeys_cmd_scanner_end:},且在它后面的内容会被 \veta{ekeys-cmd} 重新读取;
    \item \marg{ekeys-cmd arg spec}\marg{scanner code}\meta{text},同上,\meta{text} 会放在要读取的参数之前。
  \end{itemize}
  \item \texttt u\marg{alias},使用由 \cs{ekeysnewscanneralias} 设置的别名。
  \item \texttt{define}\meta{define spec},它用来在定义 \veta{ekeys-cmd} 时,进一步的自定义参数获取方式。\meta{define spec} 用法为:
  \begin{itemize}
    \item \marg{definition},它不占用任何参数。主要用作向后检查,或展开,也可以向输入中添加额外的内容。需要在 \meta{definition} 的合适位置添加 \cs{ekeys_cmd_add_args:n} 和 \cs{ekeys_cmd_scanner_end:};
    \item \marg{numbers}\marg{parameters},它占用 \meta{numbers} 个参数。根据 \meta{parameters} 向后获取参数。\meta{parameters} 的参数数量不能少于 \meta{numbers}。这是 \tn{def} 的 \veta{ekeys-cmd} 接口;
    \item \marg{numbers}\marg{parameters}\marg{definition},同上。注意:当 $\veta{numbers}\geq 0$ 时,会在 \meta{definition} 后面自动加上 \cs{ekeys_cmd_add_args:n} 和 \cs{ekeys_cmd_scanner_end:};
    \item \texttt{\{*\meta{numbers}\}}\marg{parameters}\marg{definition},\meta{numbers} 与 \meta{parameters} 没有关系。但需要在 \meta{definition} 的合适位置添加 \cs{ekeys_cmd_add_args:n} 和 \cs{ekeys_cmd_scanner_end:},且添加的参数必须和 \meta{numbers} 一致。
  \end{itemize}
  \item \texttt{lohi}\meta{defaults},它获取一组数学上下标,占用 2 个参数,第一个为下标,第二个为上标。如果不存在,则为使用默认值。设置默认值是通过给扫描器的额外的参数来设置。其中 \meta{defaults} 为:
  \begin{itemize}
    \item \marg{default_{lo}}\marg{default_{hi}},设置下标和上标的默认值;
    \item \marg{default},设置上下标的默认值;
    \item 空,表示上下标使用特殊的标记:\UseName{c_novalue_tl}。
  \end{itemize}
  它支持 \texttt\& 修饰符,即可以自动添加 \verb|_| 和 \verb|^|,但默认值不会自动添加这两个符号。
\end{itemize}

本节的这些都归类为 \veta{ekeys-cmd} 的参数扫描器,预处理器也是一类特殊的参数扫描器。
参数扫描器是完全可自定义的,所有自定义的参数处理器都需要执行 \cs{ekeys_cmd_scanner_end:} 或
与之等价的 \cs[break-at-any]{EKeysEndPreprocessor},例如在 \meta{preprocessor spec} 或 
\meta{definition} 中。

接下来的两个小节会详细介绍参数扫描器。

\section{自定义参数扫描方式}

先来介绍 \texttt{lohi} 这个预定义的 scanner。
它用来获取一组数学上下标。在找寻上下标时,会自动展开和移除空格和 \tn{relax}
(即自动移除 \BNFN{filler}),它并不需要用于数学模式,但是严格匹配的,只会匹配类别码为 7 或 8
的字符(显式或隐式的)。可以为它设置默认值。

\begin{examcode}{}
\catcode`\^=7 \catcode`\_=8
\def\temp{\relax _ \space\relax{ij}}
\DeclareEKeysCommand \faa { K{lohi} } {\detokenize{[#1|#2]}}
\DeclareEKeysCommand \fbb { K{lohi{n}} } {\detokenize{[#1|#2]}}
\DeclareEKeysCommand \fcc { K{lohi{m}{n}} } {\detokenize{[#1|#2]}}
\DeclareEKeysCommand \fmm { &p{\limits\nolimits} &K{lohi{_m}{^n}} } {\sum#1#2#3}
\obeylines
\faa ; \faa _a^b; \faa _a; \faa ^b; \faa\temp;
\fbb ; \fbb _a^b; \faa _a; \fbb ^b; \fbb\temp;
\fcc ; \fcc _a^b; \fcc _a; \fcc ^b; \fcc\temp;
$ \fmm ; \fmm _a^b; \fmm _a; \fmm ^b; \fmm\temp; \fmm\limits\temp $
\end{examcode}

\cs{ekeysnewscanneralias} 可以为扫描器全局地设置别名,在 \texttt u 中使用。
不带中间的可选参数时,可以为 scanner 设置别名;
带中间的可选参数时,可以为任意参数类型设置别名,但需写出这个类型。
\begin{examcode}{}
\catcode`\^=7 \catcode`\_=8
\ekeysnewscanneralias{lohi-xy}{lohi{x}{y}} % scanner
\ekeysnewscanneralias{lohi-K-xy}[K]{ & K{lohi{_x}{^y}} } % 任意说明符/修饰符
\DeclareEKeysCommand \fcc { K{u{lohi-xy}} } {\detokenize{[#1|#2]}}
\DeclareEKeysCommand \fmm { K{u{lohi-K-xy}} } {\sum#1#2}
\fcc ; \fcc _a^b; \fcc _a; \fcc ^b; 
$ \fmm ; \fmm _a^b; \fmm _a; \fmm ^b; $
\end{examcode}

\begin{syntax}
\V*|\ekeys_cmd_new_scanner:nnnpn| \marg{scanner} \marg{argument number} \marg{initial action} \meta{parameter list} \marg{scanner action}
\V*|\ekeys_cmd_new_scanner:nnpn| \marg{scanner} \marg{argument number} \meta{parameter list} \marg{scanner action}
\V*|\ekeys_cmd_add_args:n| \{ \marg{arg_1} \marg{arg_2} ... \marg{arg_n} \}
\V*|\ekeys_cmd_add_arg:n| \marg{arg}
\V*|\ekeys_cmd_scanner_end:|
\end{syntax}
\cs{ekeys_cmd_new_scanner:nnnpn} 用于定义一个参数扫描器。
\meta{scanner action} 中必须执行 \cs[break-at-any]{ekeys_cmd_scanner_end:},
在它后面的内容会被此扫描器之后的参数(或扫描器)读取。在 \meta{scanner action} 中用
\cs{ekeys_cmd_add_args:n} 和 \cs{ekeys_cmd_add_arg:n} 来给 \veta{ekeys-cmd} 添加参数。
还可在 \meta{initial action} 中设置 \cs{l_ekeys_cmd_scanner_args_int} 来修改 
\meta{parameter number}。添加的参数数目必须和该整数值一致。

\begin{examcode}[l3code]{}
\ExplSyntaxOn
\ekeys_cmd_new_scanner:nnpn { my/scan-bra-ket } { 2 } <#1|#2>
  {
    \ekeys_cmd_add_args:n { {#1} {#2} }
    \ekeys_cmd_scanner_end: 
  }
\ExplSyntaxOff
\DeclareEKeysCommand \foo { K{my/scan-bra-ket} } {\left<#1\middle|#2\right>}
$ \foo<a|b> $\quad $ \foo<\sum|\prod> $
\end{examcode}

针对上面这种特别简单的情况,可以直接使用 \texttt{define} scanner:
\begin{examcode}[listing only]{}
\ExplSyntaxOn
\DeclareEKeysCommand \foo { K{ define {2} {<#1|#2>} { } } } {\left<#1\middle|#2\right>}
\ExplSyntaxOff
$ \foo<a|b> $\quad $ \foo<\sum|\prod> $
\end{examcode}
甚至,由于 \meta{definition} 为空,还可以直接不写:
\begin{examcode}[listing only]{}
\ExplSyntaxOn
\DeclareEKeysCommand \foo { K{ define {2} {<#1|#2>} } } {\left<#1\middle|#2\right>}
\ExplSyntaxOff
$ \foo<a|b> $\quad $ \foo<\sum|\prod> $
\end{examcode}
这是由于,对于上面这两种情况,会在 \meta{definition} 后面自动附加合适的 
\cs{ekeys_cmd_add_args:n} 和 \cs[break-at-any]{ekeys_cmd_scanner_end:}。
如果某些情况下,不需要自动添加的 \cs{ekeys_cmd_add_args:n} 和 \cs{ekeys_cmd_scanner_end:}
该怎么办呢?答案是在参数数字前面加上 \texttt{*}:
\begin{examcode}[listing only,l3code]{}
\ExplSyntaxOn
\DeclareEKeysCommand \foo { 
  K{ define {*2} {<#1|#2>} { \ekeys_cmd_add_args:n {{#1}{#2}}\ekeys_cmd_scanner_end: } } 
} {\left<#1\middle|#2\right>}
\ExplSyntaxOff
$ \foo<a|b> $\quad $ \foo<\sum|\prod> $
\end{examcode}

当参数数目设置为 $-1$ 时,也不会自动添加 \cs{ekeys_cmd_add_args:n} 和 
\cs{ekeys_cmd_scanner_end:}。

以上是一些简单的情形,实际上参数扫描器还可以定义得更加复杂。更多的例子
见\cref{sec:vario-examples}。

参数扫描器除了可以添加参数,还可以用作检查后面的内容并进行必要的处理。针对这一类特殊的扫描器,
称之为参数预处理器,这是下一节要介绍的。

\section{参数预处理器}

\veta{document-cmd} 有“参数处理器”,而 \veta{ekeys-cmd} 有“参数\emph{预}处理器”。
“处理器”和“预处理器”的区别是,前者先按参数说明符的规则获取参数后再对其进行处理,
把结果作为实参;
而后者是对未读取的内容先一步进行处理,然后再根据参数说明符的规则获取参数作为实参。

假设我们想查看后面的那个记号是不是 \tn{relax},借助 
\cs{peek_meaning_remove:NTF} 可以这么做:
\begin{examcode}[l3code]{}
\ExplSyntaxOn
\DeclareEKeysCommand \foo { 
  K{ define { 
      \peek_meaning_remove:NTF \scan_stop: 
        { \ekeys_cmd_scanner_end: { is-relax } } 
        { \ekeys_cmd_scanner_end: { is-not-relax } }
    } 
  } m u;
} { #1. \tl_to_str:n {#2}. }
\ExplSyntaxOff \ttfamily
\foo \relax; \foo a; \foo {\relax};
\end{examcode}
可以看到,在获取第一个参数之前,我们根据后面是不是跟着 \tn{relax} 来让它获得不同的值。

这是预处理器不需要获取参数的情况,倘若需要获取参数,仍然可以用 \texttt{define} scanner:
\begin{examcode}[l3code]{}
\ExplSyntaxOn
\DeclareEKeysCommand \foo {
  K{ define {-1} { #1#2 } { 
      \tl_if_eq:nnTF {#1} {#2}
        { \ekeys_cmd_scanner_end: \BooleanTrue }
        { \ekeys_cmd_scanner_end: \BooleanFalse }
    } 
  } m 
} { \IfBooleanTF {#1} { eq } { neq } }
\ExplSyntaxOff
\foo {a}{b}; \foo {a}{a};
\end{examcode}
\verb|define{*0}{#1#2}...| 也是可行的。也可直接使用 \cs{ekeys_cmd_new_scanner:nnnpn}。

使用上面这种方式定义的预处理器通常比较简短,获取参数的方式也比较有限,若要获取更加复杂参数,
则可以使用 \texttt? scanner 和 \texttt? 预处理器指示符。
它们的区别是,预处理器指示符和参数类型的地位等同,而 \texttt? 参数扫描器则是和 \texttt{define}
扫描器等同,且它们的具体用法也不一致。

\texttt? 预处理器指示符的用法 \texttt?\marg{preprocessor action}。
\meta{preprocessor action} 通常是一个命令,它的定义中有一个 \cs{ekeys_cmd_scanner_end:}。
\begin{examcode}[l3code]{}
\ExplSyntaxOn
% 它用来合并 () 和 [] 的键值选项
\DeclareEKeysCommand \foopreprocessor { m @w{ () [] } }
  {
    \tl_if_novalue:nTF {#2}
      { 
        \tl_if_novalue:nTF {#3} 
          { \ekeys_cmd_scanner_end: { } } 
          { \ekeys_cmd_scanner_end: {#3} }
      }
      { 
        \tl_if_novalue:nTF {#3} 
          { \ekeys_cmd_scanner_end: {#2} } 
          { \ekeys_cmd_scanner_end: { #2 #1 #3 } }
      }
  }
\ExplSyntaxOff
\DeclareEKeysCommand \foo { ?{\foopreprocessor{,}} m } {\detokenize{#1}}
\foo [a=b,b=k] (c=d,e=f); \foo ;
\end{examcode}
对于上面预处理器命令中的 \cs{ekeys_cmd_scanner_end:} 可以替换为 
\cs{EKeysEndPreprocessor},它们的区别是,前者必须用在参数扫描器的定义中,
而后者若没有用在参数扫描器中,不会报错且 \texttt f-展开为空。

上例可以转换为 \texttt? scanner:
\begin{examcode}{}
\def\mergenovalue#1#2#3{%
  \IfNoValueTF{#2}{\IfNoValueTF{#3}{#1{}}{#1{#3}}}
    {\IfNoValueTF{#3}{#1{#2}}{#1{#2,#3}}}}
\DeclareEKeysCommand \foo {
  K{ ?{ @w{ () [] } }{\mergenovalue\EKeysEndPreprocessor{#1}{#2}} } m 
} {\detokenize{#1}}
\foo [a=b,b=k] (c=d,e=f); \foo ;
\end{examcode}

若 \texttt? scanner 没有给出 \meta{preprocessor spec} 没有给出 \meta{scanner code},则
会按照给定的参数说明符获取参数,并将其转化为带花括号的标准参数。
\begin{examcode}{}
\DeclareEKeysCommand \foo { K{?{ @w{[] ()} }} m m }{\detokenize{{#1}{#2}}}
\foo (brace(b)) [bracket[B]]; \foo ;
\end{examcode}

当然,参数扫描器和预处理器能做的还远不止于此,例如,使用使用 \LaTeXiii 的正则表达式库,
可以对参数进行更加细致的处理,使用 \cs{peek_regex:NTF}、\cs{peek_regex_replace_once:NnTF}
等可以对参数进行更加细致的检测。

下一小节是一些例子。它们使用了 \pkg{collectn} 宏包,它的用法可以参考文末附上的用法说明。

\section{\veta{ekeys-cmd} 诸例}\label{sec:vario-examples}

尽管 \texttt c 和 \texttt C 不能引用其它参数,但可以间接做到这一点。
并且,由于 \texttt K 和预处理器可以嵌套使用,在处理参数时是非常灵活的。
\begin{examcode}{}
\ekeysdeclarecmd \faa { 
  K{ ? {G{{3cm}}} {\def\footmpa{#1}\EKeysEndPreprocessor} }
  c{ w\footmpa } 
} {\fbox{\box#1}}
\faa \relax {你好\\ 吗? } ! % 这个 \relax 用来阻止 G 参数
%
\ekeysdeclarecmd \fbb {
  K{ ?{ 
        K{ ?{O{t} O{9ex} G{2cm}}{\EKeysEndPreprocessor{[#1][#2]{#3}}} } G{{3cm}} 
      }{\def\footmpa{#1}\EKeysEndPreprocessor} 
  }
  c { w\footmpa }
}{\fbox{\box#1}}
\fbb [c]\relax {你好\\ 吗?}!
% 这个 \relax 用来阻止内层的 G 参数,而外层的 G 参数已经由内层的预处理器给出了
\end{examcode}

嵌套使用 \texttt{K} 参数时,需要注意使用的 \cs{ekeys_cmd_scanner_end:} 次数要和需要的一致。

使用 \texttt K 类型的参数可以做到和 \tn{def} 定义宏一样的效果。
\begin{examcode}{}
\ekeysnewscanneralias{math-style}[p]
  {p{\displaystyle\textstyle\scriptstyle\scriptscriptstyle}}
\ekeysnewscanneralias{scan-bra-ket}{define{2}{<#1|#2>}}

\DeclareEKeysCommand \foo { &K{u{math-style}} K{u{scan-bra-ket}} }
  {{#1\left<#2\middle|#3\right>}}
$ \foo<a|b> $\quad $ \foo\displaystyle<\sum|\prod> $.
\end{examcode}

等价的方法是:
\begin{examcode}[listing only,l3code]{}
\ExplSyntaxOn
\ekeys_cmd_new_scanner:nnpn { my/scan-bra-ket } { 2 } <#1|#2>
  {
    \ekeys_cmd_add_args:n { {#1} {#2} }
    \ekeys_cmd_scanner_end: 
  }
\ExplSyntaxOff
\DeclareEKeysCommand \foo 
  { &p{\displaystyle\textstyle\scriptstyle\scriptscriptstyle} K{my/scan-bra-ket} } 
  {{#1\left<#2\middle|#3\right>}}
$ \foo<a|b> $\quad $ \foo\displaystyle<\sum|\prod> $.
\end{examcode}

有一类原语,它们可以通过关键字的出现与否来实现不同的行为,如 \tn{special}:
\begin{latexbnf}
\BNFItem* "\V\special" ["shipout"] <general text>
<general text> ::= <filler>"\V{\{}"<balanced text><right brace>
<filler> ::= <optional spaces> | <filler>"\V\relax"<filler>
<optional spaces> ::= <empty> | <space token><optional spaces>
\end{latexbnf}
其中,这 \BNFT{shipout} 是忽略大小写和类别码的,
而且在检测关键字是否出现时会自动展开后面的记号。\penalty-1000
\cs{collectn_scan_keyword:NTF} 为我们提供了此功能。

\begin{examcode}[l3code]{}
\ExplSyntaxOn
\newtoks \l__my_scan_nos_toks
\collectn_set_keyword:Nn \c__my_scan_nos_tl { shipout }
\ekeys_cmd_new_scanner:nnpn { my/scan-nos } { 2 }
  {
    \collectn_scan_keyword:NTF \c__my_scan_nos_tl 
      {
        \ekeys_cmd_add_arg:n { \BooleanTrue }
        \__my_scan_nos_next: 
      }
      { 
        \ekeys_cmd_add_arg:n { \BooleanFalse }
        \__my_scan_nos_next: 
      }
    % 不可将 \__my_scan_nos_next: 放到这里,
    % 否则,\collectn_scan_keyword:NTF 将会扫描到 \__my_scan_nos_next: 
  }
\cs_new:Npn \__my_scan_nos_next: 
  {
    \collectn_value:Nnw \l__my_scan_nos_toks
      {
        \ekeys_cmd_add_arg:V \l__my_scan_nos_toks 
        \ekeys_cmd_scanner_end: % <- 扫描器完成时必须使用它
      } = % 加上一个等号,这样后面就不能再出现等号了
  }
\ExplSyntaxOff

\DeclareEKeysCommand \foo { K{my/scan-nos} }
  {Is\IfBooleanF{#1}{ not} shipout, text: [#2]}
\foo {special code} \quad 
\foo ShipouT {special code} \quad 
\foo shipout\relax\bgroup special code}
\end{examcode}

不过,当字符的类别码为 1、2、10 之一时,即使它的字符码匹配,\cs{collectn_scan_keyword:NTF} 
系列命令也无法判定它为匹配的关键字,但 \TeX 原语却能识别和匹配。
同时,类别码为 13 的字符无法构成关键字。

倘若我们想要模拟 \tn{vrule} 和 \tn{hrule} 这样的命令,它们的语法规则是这样的:
\begin{latexbnf}
<vertical rule> ::= "\V\vrule"<rule specification>
<horizontal rule> ::= "\V\hrule"<rule specification>
<rule specification> ::= <optional spaces> | <rule dimension><rule specification>
<rule dimension> ::= "width"<dimen> | "height"<dimen> | "depth"<dimen>
\end{latexbnf}
像 \BNFT{width} 这个关键词,它是忽略大小写和类别码的。
而且 \BNFN{rule dimension} 需要匹配多个无序的关键词。
\cs{collectn_scan_keywords:NTF} 正好为我们提供了对应功能。
至于 \BNFN{dimen} 只需使用 \cs{collectn_value:Nnw} 即可获取。

下例定义了一个参数扫描器 \texttt{my/scan-whd},用于捕获以关键字 
\texttt{width}、\texttt{height}、\texttt{depth} 引导的长度。

\begin{examcode}[l3code]{}
\ExplSyntaxOn
\dim_new:N \l__my_scan_whd_dim % 临时保存长度
\seq_new:N \l__my_scan_whd_seq % 保存三个值,width,height,depth
\cs_generate_variant:Nn \seq_set_item:Nnn { NnV }
\collectn_set_keywords:Nn \c__my_scan_whd_tl { width, height, depth }
% 3 个参数,为 width, height, depth
\ekeys_cmd_new_scanner:nnpn { my/scan-whd } { 3 }
  {
    % initial seq
    \seq_clear:N \l__my_seq_whd_seq
    \seq_put_right:NV \l__my_seq_whd_seq \c_novalue_tl
    \seq_put_right:NV \l__my_seq_whd_seq \c_novalue_tl
    \seq_put_right:NV \l__my_seq_whd_seq \c_novalue_tl
    \__my_scan_whd_next: 
  }
\cs_new:Npn \__my_scan_whd_next: 
  {
    \collectn_scan_keywords:NTF \c__my_scan_whd_tl 
      {
        \collectn_value:Nnw \l__my_scan_whd_dim 
          {
            \seq_set_item:NnV \l__my_seq_whd_seq 
              { \l_collectn_keywords_int } \l__my_scan_whd_dim 
            \__my_scan_whd_next: 
          } = % 加上一个等号,这样后面就不能再出现等号了
      }
      { 
        \ekeys_cmd_add_args:e
          { \seq_map_function:NN \l__my_seq_whd_seq \ekeys_exp_not_braced:n }
        \ekeys_cmd_scanner_end: 
      }
  }
\ExplSyntaxOff

\DeclareEKeysCommand \foo { K{my/scan-whd} } 
  {[ W: \IfValueTF{#1}{#1}{-}; H: \IfValueTF{#2}{#2}{-}; D: \IfValueTF{#3}{#3}{-} ]}
\ttfamily
\foo height 3pt depth \dimeval{3pt+2pt-10pt} height \dimexpr 3pt-2pt wide
\end{examcode}

\begin{table}[h]
\caption{\pdfTeX、\XeTeX、\LuaTeX 可用的类别码及其含义}\label{tab:catcode}
\let\V\Verbatimize
\def\TH#1{\multicolumn{1}{c}{\normalfont\bfseries #1}}
\def\N#1 {\hbox to 1em{\hfil#1}}
\startfullpagewidth \noindent\centering
\begin{tabular}{cll>{\ttfamily}l}
\toprule
\TH{类别} & \TH{含义} & \TH{Meaning} & \TH{常见值} \\ \midrule
\N0 & 转义字符 & Escape character & \V\\ \\
\N1 & 组开始 & Beginning of group & \V\{ \\
\N2 & 组结束 & End of group & \V\} \\
\N3 & 数学切换 & Math shift & \V\$ \\
\N4 & 表对齐 & Alignment tab & \V\& \\
\N5 & 行结束 & End of line & \BNFN{return} \\
\N6 & 变量 & Parameter & \V\# \\
\N7 & 上标 & Superscript & \V\^
\\ \bottomrule
\end{tabular}\,%
\begin{tabular}{cll>{\ttfamily}l}
\toprule
\TH{类别} & \TH{含义} & \TH{Meaning} & \TH{常见值} \\ \midrule
\N8 & 下标 & Subscript & \V_ \\
\N9 & 忽略的字符 & Ignored character & \BNFN{null} \\
\N10 & 空格 & Space & \textvisiblespace \\
\N11 & 字母 & Letter & 26 个英文字母 \\
\N12 & 其它字符 & Other character & 其它 \\
\N13 & 活动字符 & Active character & \V~ \\
\N14 & 注释字符 & Comment character & \V\% \\
\N15 & 无效字符 & Invalid character & \BNFN{delete}
\\ \bottomrule
\end{tabular}\par 
\stopfullpagewidth
\footnotesize\fangsong 注:
在 \XeLaTeX 和 \LuaLaTeX 下,使用 \pkg{xeCJK} 或 \pkg{luatexja}
宏包,中日韩字符的类别码也是 11。
\end{table}

\IfGraphicsExists{cus-cn.pdf}{\section[=,float-barrier]{用法说明}\clearpage
  \usepagestyle{totalempty}
  \IterateInteger[91][1]{108}{\newpage\null\foregroundpicture[page=#1]{cus-cn.pdf}}
}{}

\end{document}
